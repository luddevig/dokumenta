\documentclass[11pt, noincludeaddress]{classes/cthit}
\usepackage{titlesec}
\usepackage{titletoc}
\usepackage{verbatimbox}

\titleformat{\paragraph}[hang]{\normalfont\normalsize\bfseries}{\theparagraph}{1em}{}
\titlespacing*{\paragraph}{0pt}{3.25ex plus 1ex minus 0.2ex}{0.7em}


\titlelabel{\S \thetitle\quad}

\graphicspath{ {images/} }

\begin{document}

\title{8-bITs Stadga}
\approved{2006--12--05}
\revisioned{2017--12--14}
\maketitle

\thispagestyle{empty}

\newpage

\makeheadfoot%

%Rubriksnivådjup
\setcounter{tocdepth}{2}
%Sidnumreringsstart
\setcounter{page}{1}
\tableofcontents

\newpage

\section{Definition}

\subsection{Benämning}

\subsubsection{Föreningens fullständiga namn}
\EIGHTBITFULL{}

\subsubsection{Föreningens akronym}
\EIGHTBIT{}

\subsubsection{Föreningens säte}
Föreningen har sitt säte i Göteborg, Västra Götaland.

\subsubsection{Föreningsform}
Föreningen är en ideell förening. Föreningen är religiöst och partipolitiskt obunden. Föreningen är ansluten till Sverok. Föreningen arbetar med likabehandling, för att alla medlemmar ska ha samma möjligheter och förutsättningar för att delta i verksamheten och påverka den. Föreningens verksamhetsår är 1 augusti till 31 juli.

\subsection{Syfte}

\subsubsection{Främjande av intresse}
Intresseföreningen har som syfte att främja intresset för TV-spel. TVspel innebär konsolspel, arkadspel och spel till handhållna (portabla) spelmaskiner.

\subsection{Rättigheter}
\subsubsection{Sektionens varumärke}
Intresseföreningen äger rätt att i namn och emblem använda
sektionens namn och symboler.


\subsection{Skyldigheter}

\subsubsection{Sektionen}
Intresseföreningen är skyldig att rätta sig efter sektionens stadga, 
reglemente och fattade beslut. 

\subsubsection{Styrelse}
Intresseföreningen måste ha en tillsatt styrelse, se {styrelse}.

\subsubsection{Årsmöten}
Intresseföreningen måste ha minst ett möte per år dit föreningens 
och sektionens medlemmar är kallade.

\subsection{Verksamhet}
\subsubsection{Ekonomi}
Intresseföreningens ekonomi skall vara fristående från sektionen.
\subsubsection{Revision}
Revisorns uppgift är att granska styrelsens arbete och redovisa det för nästa årsmöte.
Föreningen ska ha en eller två revisorer.
De som sitter i föreningens styrelse kan inte väljas till revisor.
Revisorn måste inte vara medlem i föreningen.

\newpage

\section{Medlemmar}

\subsection{Styrelse}
\label{styrelse}

\subsubsection{Sammansättning}
Intresseföreningens styrelse består av:
\begin{itemize}
        \item Ordförande
		\item Kassör
		\item Minst en extra ledamot
\end{itemize}

\subsubsection{Ansvar}
Styrelsen ansvarar för intresseföreningens medlemslista,
medlemsvärvning, sammankomster, beslut som tas på årsmötet och övrig verksamhet.

\subsubsection{Tillsättning}
Styrelsen tillsätts på årsmötet och tillträder direkt efter valet. Valbar är medlem i föreningen. En medlem kan inte
inneha mer än en post i styrelsen.


\subsection{Medlemskap}
Som medlem antas intresserad som godkänner dessa stadgar och aktivt tar ställning för ett medlemskap genom
att årligen betala föreningens medlemsavgift eller, om medlemskapet är gratis, årligen göra en skriftlig anmälan
till föreningen. Avgiftens storlek beslutas på årsmötet. Årsmötet kan besluta att det är gratis att vara medlem. En medlem som allvarligt skadar föreningen kan avstängas av styrelsen. Avstängd medlem måste diskuteras på nästa
årsmöte. Antingen så upphävs då avstängningen eller så utesluts medlemmen. Styrelsen och årsmöte kan
upphäva avstängning och uteslutning.


\subsection{Medlemsrättigheter}

\subsubsection{Närvaro-, yttrande- och förslagsrätt}
Varje medlem av intresseföreningen har närvarorätt, yttranderätt och förslagsrätt på föreningens samtliga årsmöten.

\subsubsection{Rösträtt}
Varje medlem av intresseföreningen har rösträtt i föreningens samtliga årsmöten. Alla frågor som behandlas på årsmöte eller styrelsemöte avgörs med enkel röstövervikt om inget annat står i stadgarna. Vid lika röstetal avgör slumpen.

\section{Sammankomster}

\subsection{Årsmöte}

\subsubsection{Kallelse}
Intresseföreningens styrelse beslutar om tid och plats för föreningens årsmöte.
För att mötet ska vara giltigt måste föreningens medlemmar meddelas personligen minst två veckor i förväg. Medlemmarna ska få veta tid, plats och vad som ska tas upp på mötet.
Om alla medlemmar i föreningen godkänner det kan mötet vara giltigt även om inbjudan kommit senare än två veckor innan mötet.
Mötet ska även annonseras på sektionens anslagstavla i minst två veckor i förväg.

\subsubsection{Dagordning}
Följande punkter måste behandlas på det ordinarie mötet:

\begin{itemize}
        \item Mötets öppnande
        \item Beslut om mötets giltighet
        \item Val av mötets ordförande
        \item Val av mötets sekreterare
        \item Val av två justerare tillika rösträknare
        \item Verksamhetsberättelse
        \item Ekonomisk berättelse
        \item Beslut om ansvarsfrihet för förra årets styrelse
        \item Beslut om eventuell medlemsavgift
        \item Beslut om årets verksamhetsplan
        \item Beslut om årets budget
        \item Val av årets styrelse
        \item Val av årets revisorer
        \item Övriga frågor
        \item Mötets avslutande
\end{itemize}

\subsubsection{Extra årsmöte}
Om styrelsen, revisorer eller minst hälften av föreningens medlemmar kräver det så ska styrelsen kalla till ett extra årsmöte.

\subsection{Beslut}

\subsubsection{Röstning}
Beslut genom röstning sker genom enkel majoritet.

\subsubsection{Stadgeändring}
Föreningens stadgar kan bara ändras på ett årsmöte, om förslaget på ändring står med i inbjudan till mötet. För att ändringen ska gälla, måste minst dubbelt så många rösta för ändringen som de som röstar mot.

\section{Nedläggning}
Årsmötet kan bestämma att föreningen ska läggas ned. Föreningen läggs inte ned så länge det finns minst tre medlemmar som vill driva föreningen.
Om föreningen läggs ned, ska alla föreningens skulder betalas och konton avslutas samt all lånad inventarie återlämnas till dess ägare.
Om föreningen har pengar eller saker kvar, ska dessa skänkas till en ideell förening med liknande syfte eller skickas tillbaka till Sverok.

\end{document}